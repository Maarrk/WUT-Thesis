%%%%%%%%%%%%%%%%%%%%%%%%%%%%%%%%%%%%%%%%%%%%%%%%%%%%%%%
%% Bachelor's & Master's Thesis Template             %%
%% Copyleft by Artur M. Brodzki & Piotr Woźniak      %%
%% Faculty of Electronics and Information Technology %%
%% Warsaw University of Technology, 2019-2020        %%
%%%%%%%%%%%%%%%%%%%%%%%%%%%%%%%%%%%%%%%%%%%%%%%%%%%%%%%

\documentclass[
    left=2.5cm,         % Sadly, generic margin parameter
    right=2.5cm,        % doesnt't work, as it is
    top=2.5cm,          % superseded by more specific
    bottom=3cm,         % left...bottom parameters.
    bindingoffset=6mm,  % Optional binding offset.
    nohyphenation=false % You may turn off hyphenation, if don't like.
]{eiti/eiti-thesis}

\langpol % Dla języka angielskiego mamy \langeng
\graphicspath{{img/}}             % Katalog z obrazkami.
\addbibresource{bibliografia.bib} % Plik .bib z bibliografią

\begin{document}

%--------------------------------------
% Strona tytułowa
%--------------------------------------
\MasterThesis % Dla pracy inżynierskiej mamy \EngineerThesis
\instytut{XXXXXX}
\kierunek{XXXXXX}
\specjalnosc{XXXXXX}
\przedmiot{Laboratorium Systemów Lotniczych}
\title{
    Niepotrzebnie długi i skomplikowany tytuł pracy \\
    trudny do przeczytania, zrozumienia i wymówienia
}
\engtitle{ % Tytuł po angielsku do angielskiego streszczenia
    Unnecessarily long and complicated thesis' title \\
    difficult to read, understand and pronounce
}
\author{\{Imię i Nazwisko\}}
\album{XXXXXX}
\promotor{XXXXXX}
\date{\the\year}
\maketitle

%--------------------------------------
% Streszczenie po polsku
%--------------------------------------
% \cleardoublepage % Zaczynamy od nieparzystej strony
% \streszczenie \lipsum[1-3]
% \slowakluczowe XXX, XXX, XXX

%--------------------------------------
% Streszczenie po angielsku
%--------------------------------------
% \newpage
% \abstract \kant[1-3]
% \keywords XXX, XXX, XXX

%--------------------------------------
% Oświadczenie o autorstwie
%--------------------------------------
% \cleardoublepage  % Zaczynamy od nieparzystej strony
% \pagestyle{plain}
% \makeauthorship

%--------------------------------------
% Spis treści
%--------------------------------------
% \cleardoublepage % Zaczynamy od nieparzystej strony
\newpage
\tableofcontents

%--------------------------------------
% Rozdziały
%--------------------------------------
% \cleardoublepage % Zaczynamy od nieparzystej strony
\newpage
\pagestyle{headings}

\newpage
\section{Wstęp}

\subsection{Cel ćwiczenia}
Celem ćwiczenia było praktyczne zapoznanie się z algorytmem wyznaczania pozycji w systemie nawigacji satelitarnej na przykładzie GPS.

\subsection{Wykorzystany sprzęt}

\subsection{Procedura pomiarowa}         % Wygodnie jest trzymać każdy rozdział w osobnym pliku.

%--------------------------------------------
% Literatura
%--------------------------------------------
% \cleardoublepage % Zaczynamy od nieparzystej strony
\newpage
\printbibliography

%--------------------------------------------
% Spisy (opcjonalne)
%--------------------------------------------
\newpage
\pagestyle{plain}

% Wykaz symboli i skrótów.
% Pamiętaj, żeby posortować symbole alfabetycznie
% we własnym zakresie. Ponieważ mało kto używa takiego wykazu,
% uznałem, że robienie automatycznie sortowanej listy
% na poziomie LaTeXa to za duży overkill.
% Makro \acronymlist generuje właściwy tytuł sekcji,
% w zależności od języka.
% Makro \acronym dodaje skrót/symbol do listy,
% zapewniając podstawowe formatowanie.
% //AB
\vspace{0.8cm}
\acronymlist
% \acronym{EiTI}{Wydział Elektroniki i Technik Informacyjnych}
% \acronym{PW}{Politechnika Warszawska}
% \acronym{WEIRD}{ang. \emph{Western, Educated, Industrialized, Rich and Democratic}}

\listoffigurestoc     % Spis rysunków.
\vspace{1cm}          % vertical space
\listoftablestoc      % Spis tabel.
\vspace{1cm}          % vertical space
\listofappendicestoc  % Spis załączników

% Załączniki
\newpage
\appendix{Opis kodu MATLAB}

Kod dopisany do pliku calcGpsPos.m


% \newpage
% \appendix{Nazwa załącznika 2}
% \lipsum[1-4]

% Używając powyższych spisów jako szablonu,
% możesz tu dodać swój własny wykaz bądź listę,
% np. spis algorytmów.

\end{document} % Dobranoc.
